\subsection{Гистограммы}

\subsubsection{Определение и описание}
\textit{Гистограмма}\ - функция, приближающая плотность вероятности некоторого распределения, построенная на основе выборки из него. Используются гистограммы для визуализации данных на начальном этапе статистической обработки. Построение гистограмм используется для получения эмпирической оценки плотности распределения случайной величины.

\subsubsection{Построение гистограммы}
Гистограммы строятся следующим образом: все множество значений, которые могут принимать элементы выборки, разбивается на несколько интервалов. Чаще всего, эти интервалы делают одинакового размера, но это не обязательно (в данной лабораторной работе интервалы будут одинакового размера). Если интервалы одинакового размера, то высота каждого прямоугольника гистограммы будет прямо пропорционален числу элементов выборки, попавших в этот интервал. Если же интервалы разного размера, то высота прямоугольников выбирается так, чтоб их площадь была пропорциональна числу элементов выборки, попавших в этот интервал.
