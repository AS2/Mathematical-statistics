\documentclass[14pt,a4paper,article]{ncc}
\usepackage[a4paper, mag=1000, left=2.5cm, right=1cm, top=2cm, bottom=2cm, headsep=0.7cm, footskip=1cm]{geometry}
\usepackage[utf8]{inputenc}
\usepackage[T2A]{fontenc}
\usepackage[english,russian]{babel}
\usepackage{indentfirst}
%\usepackage[dvipsnames]{xcolor}
\usepackage[colorlinks]{hyperref}
\usepackage{amsfonts} 
\usepackage{amsmath}
\usepackage{graphicx}
\usepackage{float}
\graphicspath{{figure/}}
\DeclareGraphicsExtensions{.png,.jpg}

\usepackage{fancyhdr}
\pagestyle{fancy}
\fancyhead[LE,RO]{\thepage}
\fancyfoot{}

\usepackage{listings}

\hypersetup{linkcolor=black}


\begin{document}
% Title page 
\begin{titlepage}
    \begin{center}
        \textsc{
            Санкт-Петербургский политехнический университет имени Петра Великого \\[5mm]
            Институт прикладной математики и механики\\[2mm]
            Высшая школа прикладной математики и физики            
        }   
        \vfill
        \textbf{\large
            Математическая статистика\\
            Отчёт по лабораторным работам №1-2 \\[3mm]
        }                
    \end{center}

    \vfill
    \hfill
    \begin{minipage}{0.5\textwidth}
        Выполнил: \\[2mm]   
		Студент: Сачук Александр \\
		Группа: 5030102/90201\\
    \end{minipage}

	\hfill
	\begin{minipage}{0.5\textwidth}
		Принял: \\[2mm]
		к. ф.-м. н., доцент \\   
		Баженов Андрей Николаевич
	\end{minipage}

    \vfill
    \begin{center}
        \theyear\ г.
    \end{center}
\end{titlepage}

\tableofcontents
\listoffigures
\listoftables
\newpage

\section{Постановка задачи}
\input{common/problem.tex}
\begin{enumerate}
\input{task_1/problem.tex}
\input{task_2/problem.tex}
\end{enumerate}

\section{Теория}
\subsection{Вариационный ряд}
\textit{Вариационный ряд}\ - последовательность элементов выборки, расположенных в неубывающем порядке. Одинаковые элементы повторяются.

\subsection{Выборочные числовые характеристики}
\subsubsection{Характеристики положения}
\subsubsection{Характеристики рассеяния}
\subsection{Вариационный ряд}
\textit{Вариационный ряд}\ - последовательность элементов выборки, расположенных в неубывающем порядке. Одинаковые элементы повторяются.

\subsection{Выборочные числовые характеристики}
\subsubsection{Характеристики положения}
\subsubsection{Характеристики рассеяния}
\subsection{Вариационный ряд}
\textit{Вариационный ряд}\ - последовательность элементов выборки, расположенных в неубывающем порядке. Одинаковые элементы повторяются.

\subsection{Выборочные числовые характеристики}
\subsubsection{Характеристики положения}
\subsubsection{Характеристики рассеяния}

\section{Результаты}
\subsection{Гистограммы}
\begin{itemize}
	\item{Нормальное распределение}
	\begin{figure}[H]
		\begin{center}
			\includegraphics[scale=0.333]{resources/normal10.png}
			\includegraphics[scale=0.333]{resources/normal100.png}
			\includegraphics[scale=0.333]{resources/normal1000.png}
			\caption{Гистограмма и плотность вероятности для нормального распределения [N = 10, 100, 1000]} 
		\end{center}
	\end{figure}
	
	\item{Распределение Коши}
	\begin{figure}[H]
		\begin{center}
			\includegraphics[scale=0.5]{resources/cauchy10.png}
			\includegraphics[scale=0.5]{resources/cauchy100.png}
			\includegraphics[scale=0.75]{resources/cauchy1000.png}
			\caption{Гистограмма и плотность вероятности для распределения Коши [N = 10, 100, 1000]}
		\end{center}
	\end{figure}
		
	\item{Распределение Пуассона}
	\begin{figure}[H]
		\begin{center}
			\includegraphics[scale=0.333]{resources/poisson10.png}
			\includegraphics[scale=0.333]{resources/poisson100.png}
			\includegraphics[scale=0.333]{resources/poisson1000.png}
			\caption{Гистограмма и плотность вероятности для распределения Пуассона [N = 10, 100, 1000]} 
		\end{center}
	\end{figure}
	
	\item{Равномерное распределение}
	\begin{figure}[H]
		\begin{center}
			\includegraphics[scale=0.333]{resources/uniform10.png}
			\includegraphics[scale=0.333]{resources/uniform100.png}
			\includegraphics[scale=0.333]{resources/uniform1000.png}
			\caption{Гистограмма и плотность вероятности для равномерного распределения [N = 10, 100, 1000]} 
		\end{center}
	\end{figure}

\end{itemize}
\subsection{Гистограммы}
\begin{itemize}
	\item{Нормальное распределение}
	\begin{figure}[H]
		\begin{center}
			\includegraphics[scale=0.333]{resources/normal10.png}
			\includegraphics[scale=0.333]{resources/normal100.png}
			\includegraphics[scale=0.333]{resources/normal1000.png}
			\caption{Гистограмма и плотность вероятности для нормального распределения [N = 10, 100, 1000]} 
		\end{center}
	\end{figure}
	
	\item{Распределение Коши}
	\begin{figure}[H]
		\begin{center}
			\includegraphics[scale=0.5]{resources/cauchy10.png}
			\includegraphics[scale=0.5]{resources/cauchy100.png}
			\includegraphics[scale=0.75]{resources/cauchy1000.png}
			\caption{Гистограмма и плотность вероятности для распределения Коши [N = 10, 100, 1000]}
		\end{center}
	\end{figure}
		
	\item{Распределение Пуассона}
	\begin{figure}[H]
		\begin{center}
			\includegraphics[scale=0.333]{resources/poisson10.png}
			\includegraphics[scale=0.333]{resources/poisson100.png}
			\includegraphics[scale=0.333]{resources/poisson1000.png}
			\caption{Гистограмма и плотность вероятности для распределения Пуассона [N = 10, 100, 1000]} 
		\end{center}
	\end{figure}
	
	\item{Равномерное распределение}
	\begin{figure}[H]
		\begin{center}
			\includegraphics[scale=0.333]{resources/uniform10.png}
			\includegraphics[scale=0.333]{resources/uniform100.png}
			\includegraphics[scale=0.333]{resources/uniform1000.png}
			\caption{Гистограмма и плотность вероятности для равномерного распределения [N = 10, 100, 1000]} 
		\end{center}
	\end{figure}

\end{itemize}

\section{Реализация}
Данная лабораторная работа была выполнена с использованием языка
программирования Python 3.10 в среде разработки Visual Studio Code с
использованием следующих библиотек:
\begin{itemize}
\item scipy версии 1.8.0
\item numpy версии 1.22.0
\item matplotlib версии 3.5.1
\end{itemize}


\section{Обсуждение}
\input{task_1/discuss.tex}
\input{task_2/discuss.tex}

\section{Ссылки на библиотеки}
\begin{enumerate}
\item \url{https://scipy.org/} \ - SciPy
\item \url{https://numpy.org/} \ - NumPy
\item \url{https://numpy.org/} \ - Matplotlib
\end{enumerate}

\section{Ссылки на репозиторий}
\input{common/rep.tex}

\end{document}