\documentclass[14pt,a4paper,article]{ncc}
\usepackage[a4paper, mag=1000, left=2.5cm, right=1cm, top=2cm, bottom=2cm, headsep=0.7cm, footskip=1cm]{geometry}
\usepackage[utf8]{inputenc}
\usepackage[T2A]{fontenc}
\usepackage[english,russian]{babel}
\usepackage{indentfirst}
%\usepackage[dvipsnames]{xcolor}
\usepackage[colorlinks]{hyperref}
\usepackage{amsfonts} 
\usepackage{amsmath}
\usepackage{graphicx}
\usepackage{float}
\graphicspath{{figure/}}
\DeclareGraphicsExtensions{.png,.jpg}

\usepackage{fancyhdr}
\pagestyle{fancy}
\fancyhead[LE,RO]{\thepage}
\fancyfoot{}

\usepackage{listings}

\hypersetup{linkcolor=black}


\begin{document}
% Title page 
\begin{titlepage}
    \begin{center}
        \textsc{
            Санкт-Петербургский политехнический университет имени Петра Великого \\[5mm]
            Институт прикладной математики и механики\\[2mm]
            Высшая школа прикладной математики и физики            
        }   
        \vfill
        \textbf{\large
            Математическая статистика\\
            Отчёт по лабораторным работам №1-2 \\[3mm]
        }                
    \end{center}

    \vfill
    \hfill
    \begin{minipage}{0.5\textwidth}
        Выполнил: \\[2mm]   
		Студент: Сачук Александр \\
		Группа: 5030102/90201\\
    \end{minipage}

	\hfill
	\begin{minipage}{0.5\textwidth}
		Принял: \\[2mm]
		к. ф.-м. н., доцент \\   
		Баженов Андрей Николаевич
	\end{minipage}

    \vfill
    \begin{center}
        \theyear\ г.
    \end{center}
\end{titlepage}

\tableofcontents
\listoffigures
\listoftables
\newpage

\section{Постановка задачи}
Для четырех распределений:
\begin{itemize}
	\item Нормальное распределение: $N(x, 0, 1)$
	\item Распределение Коши: $C(x, 0, 1)$
	\item Распределение Пуассона: $P(k, 10)$
	\item Равномерное распределение: $U(x, -\sqrt{3}, \sqrt{3})$
\end{itemize}
Выполнить следующие задачи:
\begin{enumerate}
Для четырех распределений:
\begin{itemize}
	\item Нормальное распределение: $N(x, 0, 1)$
	\item Распределение Коши: $C(x, 0, 1)$
	\item Распределение Пуассона: $P(k, 10)$
	\item Равномерное распределение: $U(x, -\sqrt{3}, \sqrt{3})$
\end{itemize}
Выполнить следующие задачи:
Для четырех распределений:
\begin{itemize}
	\item Нормальное распределение: $N(x, 0, 1)$
	\item Распределение Коши: $C(x, 0, 1)$
	\item Распределение Пуассона: $P(k, 10)$
	\item Равномерное распределение: $U(x, -\sqrt{3}, \sqrt{3})$
\end{itemize}
Выполнить следующие задачи:
\end{enumerate}

\section{Теория}
\subsection{Эмпирическая функция распределения}
	\subsubsection{Статический ряд}
	\noindent Статистическим ряд- последовательность различных элементов выборки $z_1, z_2, ..., z_k$ положенных в возрастающем порядке с указанием частот $n_1, n_2, ..., n_k$, с которыми эти элементы содержатся в выборке. Обычно записывается в виде таблицы.
	\subsubsection{Эмпирическая функция распределения}
	\noindent Эмпирическая (выборочная) функция распределения (э.ф.р)- относительная частота события $X<x$, полученная по данной выборке:
	    \begin{equation}
            F_n^* = P^*(X<x)
        \end{equation}
	\subsubsection{ Нахождение эмпирической функции распределения}
	\noindent Для получения относительной частоты $P^*(X<x)$  просуммируем в статистическом ряде, построенном по данной выборке, все частоты $n_i$, для некоторых элементов $z_i$ статистического ряда меньше $x$. Тогда $P^*(X<x) = \frac{1}{n}\sum_{z_i<x}n_i$. Получаем
	\begin{equation}
        F^*(x)=\frac{1}{n}\sum_{z_i<x}n_i.
    \end{equation}
    $F^*(x)$-  функция распределения дискретной случайной величины $X^*$, заданной таблицей распределения
    \begin{table}[H]
    \centering
    \begin{tabular}{|c|c|c|c|c|}
        \hline
         $X^*$&$z_1$&$z_2$&...&$z_k$\\
         \hline
         $P$&$n_1/n$&$n_2/n$&...&$n_k/n$\\
         \hline
    \end{tabular}
    \caption{Таблица распределения}
    \label{tab:my_label}
    \end{table}
    Эмпирическая функция распределения является оценкой, т. е. приближённым значением, генеральной функции распределения
    \begin{equation}
        F_n^*(x)\approx F_X(x).
    \end{equation}
	\subsection{Оценки плотности вероятности}
	\subsubsection{Определение}
	\noindent Оценкой плотности вероятности $f(x)$ называется функция $\widehat{f}(x)$, построенная на основе выборки, приближённо равная $f(x)$
    \begin{equation}
        \widehat{f}(x)\approx f(x).
    \end{equation}
	\subsubsection{Ядерные оценки}
	\noindent Представим оценки в виде суммы с числом слагаемых, равным объёму выборки:\begin{equation}
        \widehat{f}_n(x)=\frac{1}{n h_n}\sum_{i=1}^n K\left(\frac{x-x_i}{h_n}\right).
    \end{equation}
    Здесь фукнция $K(u)$, называемая ядерной (ядром), непрерывна и является плотностью вероятности, $x_1,...,x_n$ $-$ элементы выборки, ${h_n}$ — любая последовательность положительных чисел, обладающая свойствами
    \begin{equation}
        h_n\xrightarrow[n\to\infty]{}0;\;\;\;\frac{h_n}{n^-1} \xrightarrow[n\to\infty]{}\infty.
    \end{equation}
    Такие оценки называются непрерывными ядерными [2, с. 421-423].\\\\
    Гауссово (нормальное) ядро [4, с. 38]
    \begin{equation}
        K(u)=\frac{1}{\sqrt{2\pi}}e^{-\frac{u^2}{2}}.
    \end{equation}
    Правило Сильвермана [4, с. 44]
    \begin{equation}
        h_n=\left(\frac{4\hat{\sigma}^5}{3n}\right)^{1/5}\approx1.06\hat{\sigma}n^{-1/5},
    \end{equation}
    где $\hat{\sigma}$ - выборочное стандартное отклонение.
\subsection{Эмпирическая функция распределения}
	\subsubsection{Статический ряд}
	\noindent Статистическим ряд- последовательность различных элементов выборки $z_1, z_2, ..., z_k$ положенных в возрастающем порядке с указанием частот $n_1, n_2, ..., n_k$, с которыми эти элементы содержатся в выборке. Обычно записывается в виде таблицы.
	\subsubsection{Эмпирическая функция распределения}
	\noindent Эмпирическая (выборочная) функция распределения (э.ф.р)- относительная частота события $X<x$, полученная по данной выборке:
	    \begin{equation}
            F_n^* = P^*(X<x)
        \end{equation}
	\subsubsection{ Нахождение эмпирической функции распределения}
	\noindent Для получения относительной частоты $P^*(X<x)$  просуммируем в статистическом ряде, построенном по данной выборке, все частоты $n_i$, для некоторых элементов $z_i$ статистического ряда меньше $x$. Тогда $P^*(X<x) = \frac{1}{n}\sum_{z_i<x}n_i$. Получаем
	\begin{equation}
        F^*(x)=\frac{1}{n}\sum_{z_i<x}n_i.
    \end{equation}
    $F^*(x)$-  функция распределения дискретной случайной величины $X^*$, заданной таблицей распределения
    \begin{table}[H]
    \centering
    \begin{tabular}{|c|c|c|c|c|}
        \hline
         $X^*$&$z_1$&$z_2$&...&$z_k$\\
         \hline
         $P$&$n_1/n$&$n_2/n$&...&$n_k/n$\\
         \hline
    \end{tabular}
    \caption{Таблица распределения}
    \label{tab:my_label}
    \end{table}
    Эмпирическая функция распределения является оценкой, т. е. приближённым значением, генеральной функции распределения
    \begin{equation}
        F_n^*(x)\approx F_X(x).
    \end{equation}
	\subsection{Оценки плотности вероятности}
	\subsubsection{Определение}
	\noindent Оценкой плотности вероятности $f(x)$ называется функция $\widehat{f}(x)$, построенная на основе выборки, приближённо равная $f(x)$
    \begin{equation}
        \widehat{f}(x)\approx f(x).
    \end{equation}
	\subsubsection{Ядерные оценки}
	\noindent Представим оценки в виде суммы с числом слагаемых, равным объёму выборки:\begin{equation}
        \widehat{f}_n(x)=\frac{1}{n h_n}\sum_{i=1}^n K\left(\frac{x-x_i}{h_n}\right).
    \end{equation}
    Здесь фукнция $K(u)$, называемая ядерной (ядром), непрерывна и является плотностью вероятности, $x_1,...,x_n$ $-$ элементы выборки, ${h_n}$ — любая последовательность положительных чисел, обладающая свойствами
    \begin{equation}
        h_n\xrightarrow[n\to\infty]{}0;\;\;\;\frac{h_n}{n^-1} \xrightarrow[n\to\infty]{}\infty.
    \end{equation}
    Такие оценки называются непрерывными ядерными [2, с. 421-423].\\\\
    Гауссово (нормальное) ядро [4, с. 38]
    \begin{equation}
        K(u)=\frac{1}{\sqrt{2\pi}}e^{-\frac{u^2}{2}}.
    \end{equation}
    Правило Сильвермана [4, с. 44]
    \begin{equation}
        h_n=\left(\frac{4\hat{\sigma}^5}{3n}\right)^{1/5}\approx1.06\hat{\sigma}n^{-1/5},
    \end{equation}
    где $\hat{\sigma}$ - выборочное стандартное отклонение.
\subsection{Эмпирическая функция распределения}
	\subsubsection{Статический ряд}
	\noindent Статистическим ряд- последовательность различных элементов выборки $z_1, z_2, ..., z_k$ положенных в возрастающем порядке с указанием частот $n_1, n_2, ..., n_k$, с которыми эти элементы содержатся в выборке. Обычно записывается в виде таблицы.
	\subsubsection{Эмпирическая функция распределения}
	\noindent Эмпирическая (выборочная) функция распределения (э.ф.р)- относительная частота события $X<x$, полученная по данной выборке:
	    \begin{equation}
            F_n^* = P^*(X<x)
        \end{equation}
	\subsubsection{ Нахождение эмпирической функции распределения}
	\noindent Для получения относительной частоты $P^*(X<x)$  просуммируем в статистическом ряде, построенном по данной выборке, все частоты $n_i$, для некоторых элементов $z_i$ статистического ряда меньше $x$. Тогда $P^*(X<x) = \frac{1}{n}\sum_{z_i<x}n_i$. Получаем
	\begin{equation}
        F^*(x)=\frac{1}{n}\sum_{z_i<x}n_i.
    \end{equation}
    $F^*(x)$-  функция распределения дискретной случайной величины $X^*$, заданной таблицей распределения
    \begin{table}[H]
    \centering
    \begin{tabular}{|c|c|c|c|c|}
        \hline
         $X^*$&$z_1$&$z_2$&...&$z_k$\\
         \hline
         $P$&$n_1/n$&$n_2/n$&...&$n_k/n$\\
         \hline
    \end{tabular}
    \caption{Таблица распределения}
    \label{tab:my_label}
    \end{table}
    Эмпирическая функция распределения является оценкой, т. е. приближённым значением, генеральной функции распределения
    \begin{equation}
        F_n^*(x)\approx F_X(x).
    \end{equation}
	\subsection{Оценки плотности вероятности}
	\subsubsection{Определение}
	\noindent Оценкой плотности вероятности $f(x)$ называется функция $\widehat{f}(x)$, построенная на основе выборки, приближённо равная $f(x)$
    \begin{equation}
        \widehat{f}(x)\approx f(x).
    \end{equation}
	\subsubsection{Ядерные оценки}
	\noindent Представим оценки в виде суммы с числом слагаемых, равным объёму выборки:\begin{equation}
        \widehat{f}_n(x)=\frac{1}{n h_n}\sum_{i=1}^n K\left(\frac{x-x_i}{h_n}\right).
    \end{equation}
    Здесь фукнция $K(u)$, называемая ядерной (ядром), непрерывна и является плотностью вероятности, $x_1,...,x_n$ $-$ элементы выборки, ${h_n}$ — любая последовательность положительных чисел, обладающая свойствами
    \begin{equation}
        h_n\xrightarrow[n\to\infty]{}0;\;\;\;\frac{h_n}{n^-1} \xrightarrow[n\to\infty]{}\infty.
    \end{equation}
    Такие оценки называются непрерывными ядерными [2, с. 421-423].\\\\
    Гауссово (нормальное) ядро [4, с. 38]
    \begin{equation}
        K(u)=\frac{1}{\sqrt{2\pi}}e^{-\frac{u^2}{2}}.
    \end{equation}
    Правило Сильвермана [4, с. 44]
    \begin{equation}
        h_n=\left(\frac{4\hat{\sigma}^5}{3n}\right)^{1/5}\approx1.06\hat{\sigma}n^{-1/5},
    \end{equation}
    где $\hat{\sigma}$ - выборочное стандартное отклонение.

\section{Результаты}
\subsection{Характеристики положения и рассеяния}
\noindent Как было проведено округление:\\
В оценке $x=E  \pm D$ вариации подлежит первая цифра после точки. В данном случае $x=0.0 \pm 0.1k$,  $k$ - зависит от доверительной вероятности и вида распределения (рассматривается в дальнейшем цикле лабораторных работ). Округление сделано для  $k=1$.
	\begin{table}[H]
		\centering
		\begin{tabular}[t]{|l|r|r|r|r|r|}
			\hline
			Characteristic   &      Mean &    Median &       $z_R$ &      $z_Q$ &      $z_{tr}$ \\
			\hline
			Normal E(z) 10 & 0.011104 & 0.008388 & 0.007869 & 0.332015 & 0.291675 \\
\hline
Normal D(z) 10 & 0.095662 & 0.129384 & 0.181631 & 0.126743 & 0.113728 \\
\hline
E(z) \pm \sqrt{D(z)} & [-0.298189; & [-0.351312; & [-0.418313; & [-0.023995; & [-0.045561; \\ & 0.320397] & 0.368088] & 0.434051] & 0.688025] & 0.628911] \\
\hline
Normal E(z) 100 & -0.002418 & -0.007771 & -0.010427 & 0.012414 & 0.021683 \\
\hline
Normal D(z) 100 & 0.011254 & 0.017346 & 0.094536 & 0.013636 & 0.013397 \\
\hline
E(z) \pm \sqrt{D(z)} & [-0.108503; & [-0.139475; & [-0.317894; & [-0.104359; & [-0.094062; \\ & 0.103667] & 0.123933] & 0.29704] & 0.129187] & 0.137428] \\
\hline
Normal E(z) 1000 & 0.000378 & -0.00027 & -0.002986 & 0.001661 & 0.002754 \\
\hline
Normal D(z) 1000 & 0.000998 & 0.001538 & 0.062421 & 0.001223 & 0.001163 \\
\hline
E(z) \pm \sqrt{D(z)} & [-0.031213; & [-0.039487; & [-0.252828; & [-0.03331; & [-0.031349; \\ 
 & 0.031969] & 0.038947] & 0.246856] & 0.036632] & 0.036857] \\
\hline
		\end{tabular}
		\caption{Нормальное распределение}
		\label{tab:normal}
	\end{table}
	
	\begin{table}[H]
	\centering
		\begin{tabular}[t]{|l|r|r|r|r|r|}
			\hline
			Characteristic   &        Mean &    Median &            $z_R$ &       $z_Q$ &      $z_{tr}$ \\
			\hline
			Cauchy E(z) 10 & 9.529885 & 0.030773 & 47.393832 & 1.3084 & 0.782301 \\
\hline
Cauchy D(z) 10 & 116013.195536 & 0.305832 & 2898167.173177 & 10.712325 & 1.88495 \\
\hline
E(z) \pm \sqrt{D(z)} & [-331.077213; & [-0.522248; & [-1655.006583; & [-1.964569; & [-0.590634; \\
 & 350.136983] & 0.583794] & 1749.794247] & 4.581369] & 2.155236] \\
\hline
Cauchy E(z) 100 & -0.502032 & 0.007131 & -23.286978 & 0.043542 & 0.050521 \\
\hline
Cauchy D(z) 100 & 1018.714484 & 0.023291 & 2513115.181017 & 0.051077 & 0.025707 \\
\hline
E(z) \pm \sqrt{D(z)} & [-32.419339; & [-0.145483; & [-1608.567767; & [-0.18246; & [-0.109813; \\
 & 31.415275] & 0.159745] & 1561.993811] & 0.269544] & 0.210855] \\
\hline
Cauchy E(z) 1000 & -1.856742 & 0.001006 & -942.850162 & 0.005274 & 0.005073 \\
\hline
Cauchy D(z) 1000 & 13277.329557 & 0.002457 & 3317087798.656605 & 0.004868 & 0.002564 \\      
\hline
E(z) \pm \sqrt{D(z)} & [-117.084037; & [-0.048562; & [-58537.014785; & [-0.064497; & [-0.045563; \\
 & 113.370553] & 0.050574] & 56651.314461] & 0.075045] & 0.055709] \\
\hline
		\end{tabular}
	\caption{Распределение Коши}
	\label{tab:cauchy}
	\end{table}

\begin{table}[H]
		\centering
		\begin{tabular}[t]{|l|r|r|r|r|r|}
			\hline
			Characteristic    &      Mean &   Median &       $z_R$ &      $z_Q$ &     $z_{tr}$ \\
			\hline
			Poisson E(z) 10 & 9.9783 & 9.8555 & 10.249 & 10.913 & 10.765167 \\
\hline
Poisson D(z) 10 & 1.049619 & 1.44237 & 1.823999 & 1.480431 & 1.31727 \\
\hline
E(z) \pm \sqrt{D(z)} & [8.953791; & [8.654513; & [8.898445; & [9.69627; & [9.617443; \\      
 & 11.002809] & 11.056487] & 11.599555] & 12.12973] & 11.912891] \\
\hline
Poisson E(z) 100 & 10.01941 & 9.848 & 11.0115 & 9.9795 & 9.9588 \\
\hline
Poisson D(z) 100 & 0.1066 & 0.224396 & 1.015618 & 0.15533 & 0.132514 \\
\hline
E(z) \pm \sqrt{D(z)} & [9.692913; & [9.374295; & [10.003721; & [9.585381; & [9.594775; \\    
 & 10.345907] & 10.321705] & 12.019279] & 10.373619] & 10.322825] \\
\hline
Poisson E(z) 1000 & 9.994977 & 9.9965 & 11.644 & 9.9935 & 9.864228 \\
\hline
Poisson D(z) 1000 & 0.009557 & 0.003238 & 0.691764 & 0.003208 & 0.011038 \\
\hline
		\end{tabular}
		
		\caption{Распределение Пуассона}
		\label{tab:poisson}
	\end{table}

\begin{table}[H]
		\centering
		\begin{tabular}[t]{|l|r|r|r|r|r|}
			\hline
			Characteristic    &      Mean &    Median &       $z_{R}$ &       $z_Q$ &      $z_{tr}$ \\
			\hline
			Uniform E(z) 10 & 0.022043 & 0.016613 & 0.015673 & 0.340336 & 0.33759 \\
\hline
Uniform D(z) 10 & 0.099705 & 0.231529 & 0.043987 & 0.122888 & 0.151837 \\
\hline
E(z) \pm \sqrt{D(z)} & [-0.293718; & [-0.464562; & [-0.194058; & [-0.010218; & [-0.052073; \\ & 0.337804] & 0.497788] & 0.225404] & 0.69089] & 0.727253] \\
\hline
Uniform E(z) 100 & 0.005039 & 0.0059 & 0.001225 & 0.022797 & 0.039371 \\
\hline
Uniform D(z) 100 & 0.010492 & 0.030199 & 0.000583 & 0.015156 & 0.020264 \\
\hline
E(z) \pm \sqrt{D(z)} & [-0.097391; & [-0.167879; & [-0.02292; & [-0.100313; & [-0.102981; \\ 
 & 0.107469] & 0.179679] & 0.02537] & 0.145907] & 0.181723] \\
\hline
Uniform E(z) 1000 & 0.00031 & 0.001299 & -0.000103 & 0.002594 & 0.004221 \\
\hline
Uniform D(z) 1000 & 0.00103 & 0.002934 & 5e-06 & 0.001581 & 0.001991 \\
\hline
E(z) \pm \sqrt{D(z)} & [-0.031784; & [-0.052867; & [-0.002339; & [-0.037168; & [-0.0404; \\  
 & 0.032404] & 0.055465] & 0.002133] & 0.042356] & 0.048842] \\
\hline
		\end{tabular}
		\caption{Равномерное распределение}
		\label{tab:uniform}
	\end{table}
\subsection{Характеристики положения и рассеяния}
\noindent Как было проведено округление:\\
В оценке $x=E  \pm D$ вариации подлежит первая цифра после точки. В данном случае $x=0.0 \pm 0.1k$,  $k$ - зависит от доверительной вероятности и вида распределения (рассматривается в дальнейшем цикле лабораторных работ). Округление сделано для  $k=1$.
	\begin{table}[H]
		\centering
		\begin{tabular}[t]{|l|r|r|r|r|r|}
			\hline
			Characteristic   &      Mean &    Median &       $z_R$ &      $z_Q$ &      $z_{tr}$ \\
			\hline
			Normal E(z) 10 & 0.011104 & 0.008388 & 0.007869 & 0.332015 & 0.291675 \\
\hline
Normal D(z) 10 & 0.095662 & 0.129384 & 0.181631 & 0.126743 & 0.113728 \\
\hline
E(z) \pm \sqrt{D(z)} & [-0.298189; & [-0.351312; & [-0.418313; & [-0.023995; & [-0.045561; \\ & 0.320397] & 0.368088] & 0.434051] & 0.688025] & 0.628911] \\
\hline
Normal E(z) 100 & -0.002418 & -0.007771 & -0.010427 & 0.012414 & 0.021683 \\
\hline
Normal D(z) 100 & 0.011254 & 0.017346 & 0.094536 & 0.013636 & 0.013397 \\
\hline
E(z) \pm \sqrt{D(z)} & [-0.108503; & [-0.139475; & [-0.317894; & [-0.104359; & [-0.094062; \\ & 0.103667] & 0.123933] & 0.29704] & 0.129187] & 0.137428] \\
\hline
Normal E(z) 1000 & 0.000378 & -0.00027 & -0.002986 & 0.001661 & 0.002754 \\
\hline
Normal D(z) 1000 & 0.000998 & 0.001538 & 0.062421 & 0.001223 & 0.001163 \\
\hline
E(z) \pm \sqrt{D(z)} & [-0.031213; & [-0.039487; & [-0.252828; & [-0.03331; & [-0.031349; \\ 
 & 0.031969] & 0.038947] & 0.246856] & 0.036632] & 0.036857] \\
\hline
		\end{tabular}
		\caption{Нормальное распределение}
		\label{tab:normal}
	\end{table}
	
	\begin{table}[H]
	\centering
		\begin{tabular}[t]{|l|r|r|r|r|r|}
			\hline
			Characteristic   &        Mean &    Median &            $z_R$ &       $z_Q$ &      $z_{tr}$ \\
			\hline
			Cauchy E(z) 10 & 9.529885 & 0.030773 & 47.393832 & 1.3084 & 0.782301 \\
\hline
Cauchy D(z) 10 & 116013.195536 & 0.305832 & 2898167.173177 & 10.712325 & 1.88495 \\
\hline
E(z) \pm \sqrt{D(z)} & [-331.077213; & [-0.522248; & [-1655.006583; & [-1.964569; & [-0.590634; \\
 & 350.136983] & 0.583794] & 1749.794247] & 4.581369] & 2.155236] \\
\hline
Cauchy E(z) 100 & -0.502032 & 0.007131 & -23.286978 & 0.043542 & 0.050521 \\
\hline
Cauchy D(z) 100 & 1018.714484 & 0.023291 & 2513115.181017 & 0.051077 & 0.025707 \\
\hline
E(z) \pm \sqrt{D(z)} & [-32.419339; & [-0.145483; & [-1608.567767; & [-0.18246; & [-0.109813; \\
 & 31.415275] & 0.159745] & 1561.993811] & 0.269544] & 0.210855] \\
\hline
Cauchy E(z) 1000 & -1.856742 & 0.001006 & -942.850162 & 0.005274 & 0.005073 \\
\hline
Cauchy D(z) 1000 & 13277.329557 & 0.002457 & 3317087798.656605 & 0.004868 & 0.002564 \\      
\hline
E(z) \pm \sqrt{D(z)} & [-117.084037; & [-0.048562; & [-58537.014785; & [-0.064497; & [-0.045563; \\
 & 113.370553] & 0.050574] & 56651.314461] & 0.075045] & 0.055709] \\
\hline
		\end{tabular}
	\caption{Распределение Коши}
	\label{tab:cauchy}
	\end{table}

\begin{table}[H]
		\centering
		\begin{tabular}[t]{|l|r|r|r|r|r|}
			\hline
			Characteristic    &      Mean &   Median &       $z_R$ &      $z_Q$ &     $z_{tr}$ \\
			\hline
			Poisson E(z) 10 & 9.9783 & 9.8555 & 10.249 & 10.913 & 10.765167 \\
\hline
Poisson D(z) 10 & 1.049619 & 1.44237 & 1.823999 & 1.480431 & 1.31727 \\
\hline
E(z) \pm \sqrt{D(z)} & [8.953791; & [8.654513; & [8.898445; & [9.69627; & [9.617443; \\      
 & 11.002809] & 11.056487] & 11.599555] & 12.12973] & 11.912891] \\
\hline
Poisson E(z) 100 & 10.01941 & 9.848 & 11.0115 & 9.9795 & 9.9588 \\
\hline
Poisson D(z) 100 & 0.1066 & 0.224396 & 1.015618 & 0.15533 & 0.132514 \\
\hline
E(z) \pm \sqrt{D(z)} & [9.692913; & [9.374295; & [10.003721; & [9.585381; & [9.594775; \\    
 & 10.345907] & 10.321705] & 12.019279] & 10.373619] & 10.322825] \\
\hline
Poisson E(z) 1000 & 9.994977 & 9.9965 & 11.644 & 9.9935 & 9.864228 \\
\hline
Poisson D(z) 1000 & 0.009557 & 0.003238 & 0.691764 & 0.003208 & 0.011038 \\
\hline
		\end{tabular}
		
		\caption{Распределение Пуассона}
		\label{tab:poisson}
	\end{table}

\begin{table}[H]
		\centering
		\begin{tabular}[t]{|l|r|r|r|r|r|}
			\hline
			Characteristic    &      Mean &    Median &       $z_{R}$ &       $z_Q$ &      $z_{tr}$ \\
			\hline
			Uniform E(z) 10 & 0.022043 & 0.016613 & 0.015673 & 0.340336 & 0.33759 \\
\hline
Uniform D(z) 10 & 0.099705 & 0.231529 & 0.043987 & 0.122888 & 0.151837 \\
\hline
E(z) \pm \sqrt{D(z)} & [-0.293718; & [-0.464562; & [-0.194058; & [-0.010218; & [-0.052073; \\ & 0.337804] & 0.497788] & 0.225404] & 0.69089] & 0.727253] \\
\hline
Uniform E(z) 100 & 0.005039 & 0.0059 & 0.001225 & 0.022797 & 0.039371 \\
\hline
Uniform D(z) 100 & 0.010492 & 0.030199 & 0.000583 & 0.015156 & 0.020264 \\
\hline
E(z) \pm \sqrt{D(z)} & [-0.097391; & [-0.167879; & [-0.02292; & [-0.100313; & [-0.102981; \\ 
 & 0.107469] & 0.179679] & 0.02537] & 0.145907] & 0.181723] \\
\hline
Uniform E(z) 1000 & 0.00031 & 0.001299 & -0.000103 & 0.002594 & 0.004221 \\
\hline
Uniform D(z) 1000 & 0.00103 & 0.002934 & 5e-06 & 0.001581 & 0.001991 \\
\hline
E(z) \pm \sqrt{D(z)} & [-0.031784; & [-0.052867; & [-0.002339; & [-0.037168; & [-0.0404; \\  
 & 0.032404] & 0.055465] & 0.002133] & 0.042356] & 0.048842] \\
\hline
		\end{tabular}
		\caption{Равномерное распределение}
		\label{tab:uniform}
	\end{table}

\section{Реализация}
Данная лабораторная работа была выполнена с использованием языка
программирования Python 3.10 в среде разработки Visual Studio Code с
использованием следующих библиотек:
\begin{itemize}
\item scipy версии 1.8.0
\item numpy версии 1.22.0
\item matplotlib версии 3.5.1
\end{itemize}


\section{Обсуждение}
\subsection{Гистограммы}
Полученные результаты работы говорят о том, что при увеличении размеров выборок, гистограммы все ближе к графику плотности вероятности того закона, по которому были сгенерированы элементы выборок. Верно и обратно: чем меньше выборка, тем хуже по ней можно определить закон, по которой эта выборка генерировалась. 

Также одним из ключевых выводов является тот факт, что по
маленькому размеру выборки (n = 10) очень трудно отличить гистограммы, а,
следовательно, и определить закон, по которой генерировалась выборка. Действительно, гистограмма выборки, построенной по распределению Пуассона при n = 10, могла бы с тем же успехом описывать график равномерного распределения (если не учитывать один единственный всплеск гистограммы, который вообще мог остаться незамеченным при более
широких интервалах боксов гистограммы).

При выборках n = 1000 видно, что гистограммы уже достаточно
неплохо приближаются к графикам плотностей соответствующих законов
распределения: в равномерном распределении отклонения гистограммы от
графика незначительны, а в нормальном распределении уже наблюдаются
«хвосты», которые позволяют отличить треугольное распределение от
нормального.
\subsection{Гистограммы}
Полученные результаты работы говорят о том, что при увеличении размеров выборок, гистограммы все ближе к графику плотности вероятности того закона, по которому были сгенерированы элементы выборок. Верно и обратно: чем меньше выборка, тем хуже по ней можно определить закон, по которой эта выборка генерировалась. 

Также одним из ключевых выводов является тот факт, что по
маленькому размеру выборки (n = 10) очень трудно отличить гистограммы, а,
следовательно, и определить закон, по которой генерировалась выборка. Действительно, гистограмма выборки, построенной по распределению Пуассона при n = 10, могла бы с тем же успехом описывать график равномерного распределения (если не учитывать один единственный всплеск гистограммы, который вообще мог остаться незамеченным при более
широких интервалах боксов гистограммы).

При выборках n = 1000 видно, что гистограммы уже достаточно
неплохо приближаются к графикам плотностей соответствующих законов
распределения: в равномерном распределении отклонения гистограммы от
графика незначительны, а в нормальном распределении уже наблюдаются
«хвосты», которые позволяют отличить треугольное распределение от
нормального.

\section{Ссылки на библиотеки}
\begin{enumerate}
\item \url{https://scipy.org/} \ - SciPy
\item \url{https://numpy.org/} \ - NumPy
\item \url{https://numpy.org/} \ - Matplotlib
\item \url{https://seaborn.pydata.org/} \ - Seaborn
\end{enumerate}

\section{Ссылки на репозиторий}
\url{https://github.com/AS2/Mathematical-statistics} \ - GitHub репозиторий

\end{document}